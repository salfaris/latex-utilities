\documentclass[11pt,oneside]{article}

\usepackage{amssymb, amsmath, geometry}
\usepackage{parskip}

% Nice colored boxes.
\usepackage[dvipsnames]{xcolor}
\usepackage[many]{tcolorbox}

% Box 1
\newtcolorbox{myAwesomeBox}{
    enhanced,
    sharp corners,
    breakable,  % Allows page break.
    borderline west={2pt}{0pt}{Red},
    colback=Red!10,  % Background color.
    colframe=Red!10  % Frame (border) color.
}

% Box 2
\newtcolorbox{anotherAwesomeBox}{
    enhanced,
    sharp corners,
    breakable,  % Allows page break.
    borderline west={2pt}{0pt}{Cerulean},
    colback=Cerulean!10,  % Background color.
    colframe=Cerulean!10  % Frame (border) color.
}

% Box 3
\newtcolorbox{proofBox}{
    enhanced,
    sharp corners,
    breakable,  % Allows page break.
    borderline west={2pt}{0pt}{Gray},
    colback=Gray!10,  % Background color.
    colframe=Gray!10  % Frame (border) color.
}

\begin{document} 

Here is our pastel color box theorem.

\begin{myAwesomeBox}
    \textbf{Theorem A.} There exists a nice pastel-colored blue box.
\end{myAwesomeBox}

\begin{anotherAwesomeBox}
    \textit{Proof}. This is a nice pastel-colored blue box. $\blacksquare$
\end{anotherAwesomeBox}

\medskip

Let's try an actual theorem which has some use in number theory.

\begin{myAwesomeBox}
    \textbf{Proposition B.} Consider a weakly modular function $ f: \mathcal{H} \to \mathbb{C} $ of weight $ k $ with respect to $ \Gamma $ . Then $ f $ is periodic of some period $ h $ and there exists a function $ g: \mathcal{D^*} \to \mathbb{C} $ such that $ f(z) = g(q_h) $ where $ q_h(z) = e^{2 \pi i z/h} $.
\end{myAwesomeBox}

\begin{proofBox}
    \textit{Proof.} It should be clear that any congruence subgroup $ \Gamma \subseteq \mathrm{SL}_2(\mathbb{Z}) $ contains a translation matrix of the form
    \[
       \gamma = \begin{pmatrix}
           1 & h \\
           0 & 1
       \end{pmatrix}.
    \]
    Computing the factor of automorphy under $ \gamma $, we have $ j(\gamma, z) = 1 $ for any $ z \in \mathcal{H} $ which, by weakly modularity of $ f $, implies that
    \[
       f(z) = f[\gamma]_k = f(\gamma(z)) = f(z + h).
    \]
    In other words, $ f $ is periodic of period $ h $. It is obvious that the function $ q_h $ is a holomorphic function $ \mathcal{H} \to \mathcal{D^*} $ and is periodic of period $ h $ as well. So we can consider the function $ g: \mathcal{D^*} \to \mathbb{C} $ defined by
    \[
       g(q_h) = f \left( \frac{h \log q_h}{2 \pi i} \right),
    \]
    which satisfies $ f(z) = g(q_h) $. Since $ f $ is periodic of period $ h $, we can choose any branch of $ \log q_h $ in $ \mathcal{H} $ and so $ g $ is well-defined. This completes the proof. $ \blacksquare$
\end{proofBox}

\end{document}